\documentclass[a4paper,11pt]{scrartcl}
\usepackage[utf8]{inputenc}
\usepackage[english]{babel}

\usepackage[headsepline]{scrlayer-scrpage}
\ohead{Bernd Schwarzenbacher}
\ihead{UE5}

\usepackage{amsmath}
\usepackage{amssymb}
\usepackage{commath}
\usepackage{mathtools}
\usepackage[retainorgcmds]{IEEEtrantools}

\usepackage{enumitem}

\newcommand*{\dx}{\dif{}x}
\newcommand*{\ds}{\dif{}s_x}
\newcommand*{\IO}[1]{\int_{\Omega(t)}{#1}\dx}
\newcommand*{\IBO}[1]{\int_{\partial\Omega(t)}{#1}\ds}
\newcommand*{\hs}{\widehat{\sigma}}


\DeclareMathOperator*{\Div}{Div}
\DeclareMathOperator*{\spur}{spur}

\begin{document}

\begin{enumerate}[label*=\textbf{5.\arabic*.}]
  \item

  \begin{enumerate}[label=\textbf{\alph*)}]
    \item
      Partiell integrieren pro Zeile $i$ von $\sigma$:
      \[\IO{(a \times x) \nabla \cdot \sigma_i} = - \IO{\nabla(a\times x)
        \cdot \sigma_i} + \IBO{(a\times x) \sigma_i n}\]
    \item
      \begin{IEEEeqnarray*}{Cl}
        &\IO{\sum_{i,j=1}^3 \partial_t (a\times x)_i \sigma_{ij}} \\
        =& a \cdot
        \IBO{x \times {(\sigma n)}} -\IO{(a\times x) \cdot (\nabla \cdot \sigma)} \\
        =& a \cdot \IBO{x \times \left( \partial_t{\rho v} + \nabla \cdot (\rho
            v v^T)\right) - x \times (\rho f)}-\IO{(a\times x) \cdot (\nabla \cdot \sigma)} \\
        =& 0 \quad \forall a, \forall \Omega
      \end{IEEEeqnarray*}

    \item
      Aus der Gleichung $\sum_{i,j=1}^3 \partial_j (a\times x)_i \sigma_{ij}$
      wollen wir die Regeln $\sigma_{ij} - \sigma_{ji} = 0$ für $i\neq j$ folgern.

      Wir sehen aus
      \[
        (a \times x) =
      \begin{pmatrix}
        a_2x_3 - a_3 x_2 \\ a_3 x_1 - a_1 x_3 \\ a_1 x_2 - a_2 x_1
      \end{pmatrix}
      \]
      , dass $\partial_j (a\times x)_i = \pm a_k$ für $i \neq j \neq k$ und dass
      es sich mit den Vorzeichen auch ausgeht.
  \end{enumerate}

  \item
    \[\nabla\cdot(p v) = \nabla p \cdot v + p \underbrace{(\nabla \cdot v)}_{=0}\]
    \[\partial_t \frac{1}{2}\rho \abs{v}^2 = \frac{1}{2} \rho_t \abs{v}^2 +
      \rho v \cdot v_t\]
    \[\Div(\rho v v^T) \cdot v = v \nabla \cdot (\rho v) \cdot v + \rho (v \cdot
      \nabla) v \cdot v = -
      \abs{v}^2 \rho_t + \rho (v \cdot \nabla) \abs{v}^2\]
    \[\partial_t(\rho v) \cdot v = \rho_t \abs{v}^2 + \rho v_t \cdot v \]
    \[\nabla \cdot \left(\frac{1}{2} \rho \abs{v}^2 v\right) =
      \frac{1}{2}\nabla \rho \cdot \abs{v}^2 v + \rho \frac{1}{2} \nabla \cdot (\abs{v}^2 v)\]
    \[
      =\frac{1}{2}\nabla \rho \cdot \abs{v}^2 v + \rho \frac{1}{2} (\nabla
      \abs{v}^2 \cdot v + \abs{v}^2 \underbrace{\nabla \cdot v}_{=0})
    \]
    \[
      =\frac{1}{2}\nabla \rho \cdot \abs{v}^2 v + \rho (v \cdot \nabla) v \cdot v
    \]
    \begin{IEEEeqnarray*}{rCl}
\partial_t (\rho v) + \Div(\rho v v^T) + \nabla p &=& \rho f \\
\partial_t (\rho v) \cdot v + \Div(\rho v v^T) \cdot v + \nabla p \cdot v &=&
\rho f \cdot v \\
\partial_t (\rho v) \cdot v + \Div(\rho v v^T) \cdot v + \nabla \cdot (p v) &=&
\rho f \cdot v \\
    \end{IEEEeqnarray*}

  \item
    \[Q(t)^T = (e^{-t W})^T = e^{-t W^T} = e^{t W} \Rightarrow Q(t) Q(t)^T = I\]
    Da schiefsymmetrische Matrizen auf der Diagonale nur $0$ Einträge haben gilt:
    \[\det{Q(t)} = \det{e^{-tW}} = e^{\spur({-tW})} = e^0 = 1\]
    \[\partial_t Q(t) = -W Q(t)\]
    Aus
    \[\hs(-W Q(t) Q(t)^T + Q(t) A Q(t)^T) = Q(t) \hs(A) Q(t)^T\]
    folgt für $t=0$:
    \[\hs(-W + A) = \hs(A)\]
    Wählt man nun den schiefsymmetrischen Anteil von $A$ für $W = \frac{1}{2} (A
    - A^T)$
    folgt die Behauptung.

  \item

\end{enumerate}

\end{document}