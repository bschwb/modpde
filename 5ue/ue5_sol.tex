\documentclass[a4paper,11pt]{scrartcl}
\usepackage[utf8]{inputenc}
\usepackage[english]{babel}

\usepackage[headsepline]{scrlayer-scrpage}
\ohead{Bernd Schwarzenbacher}
\ihead{UE5}

\usepackage{amsmath}
\usepackage{amssymb}
\usepackage{commath}
\usepackage{mathtools}
\usepackage[retainorgcmds]{IEEEtrantools}

\usepackage{enumitem}

\newcommand*{\dx}{\dif{}x}
\newcommand*{\ds}{\dif{}s_x}
\newcommand*{\IO}[1]{\int_{\Omega(t)}{#1}\dx}}
\newcommand*{\IBO}[1]{\int_{\partial\Omega(t)}{#1}\ds}}

\begin{document}

\begin{enumerate}[label*=\textbf{5.\arabic*.}]
  \item

  \begin{enumerate}[label=\textbf{\alph*)}]
    \item
      Partiell integrieren pro Zeile $i$ von $\sigma$:
      \[\IO{(a \times x) \nabla \cdot \sigma_i}} = - \IO{\nabla(a\times x)
        \cdot \sigma_i} + \IBO{(a\times x) \sigma_i n}\]
    \item
      \begin{IEEEeqnarray*}{Cl}
        &\IO{\sum_{i,j=1}^3 \partial_t (a\times x)_i \sigma_{ij}} \\
        =& a \cdot
        \IBO{x \times {(\sigma n)}} -\IO{(a\times x) \cdot (\nabla \cdot \sigma)} \\
        =& a \cdot \IBO{x \times \left( \partial_t{\rho v} + \nabla \cdot (\rho
            v v^T)\right) - x \times (\rho f)}-\IO{(a\times x) \cdot (\nabla \cdot \sigma)} \\
        =& 0 \quad \forall a, \forall \Omega
      \end{IEEEeqnarray*}

    \item
      Aus der Gleichung $\sum_{i,j=1}^3 \partial_j (a\times x)_i \sigma_{ij}$
      wollen wir die Regeln $\sigma_{ij} - \sigma_{ji} = 0$ für $i\neq j$ folgern.

      Wir sehen aus
      \[
        (a \times x) =
      \begin{pmatrix}
        a_2x_3 - a_3 x_2 \\ a_3 x_1 - a_1 x_3 \\ a_1 x_2 - a_2 x_1
      \end{pmatrix}
      \]
      , dass $\partial_j (a\times x)_i = \pm a_k$ für $i \neq j \neq k$ und dass
      es sich mit den Vorzeichen auch ausgeht.
  \end{enumerate}

  \item
  \item
  \item
\end{enumerate}

\end{document}