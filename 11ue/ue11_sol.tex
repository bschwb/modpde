\documentclass[a4paper,11pt]{scrartcl}
\usepackage[ngerman]{babel}
\usepackage[utf8]{inputenc}

\usepackage[headsepline]{scrlayer-scrpage}
\ohead{Bernd Schwarzenbacher}
\ihead{UE11}

\usepackage{amssymb}
\usepackage{amsmath}
\usepackage{commath}
\usepackage{mathtools}
\usepackage{dsfont}
\usepackage[retainorgcmds]{IEEEtrantools}

\usepackage{enumitem}

\newcommand*{\eps}{\varepsilon}
\newcommand*{\Ld}{\mathcal{O}}
\newcommand*{\dx}{\dif{}x}
\newcommand*{\indn}{\mathds{1}_{\left(0, \frac{1}{n}\right)}}
\newcommand*{\iW}{\mathds{1}_W}

\setlist[enumerate,2]{label=\textbf{\alph*)}}

\begin{document}
\begin{enumerate}[label*=\textbf{11.\arabic*.}]

% ==================== 11.1 ====================
\item
  \begin{enumerate}
  \item
  \item
  \item
  \end{enumerate}


% ==================== 11.2 ====================
\item
  \begin{enumerate}
  \item
  \item
  \end{enumerate}


% ==================== 11.3 ====================
\item
  Mit $W = a + [0, \eps)^d$ gilt
  \[ \int_W g_\eps(x) \dx  = \int_Y g(x + a) \eps^d \dx = \eps^d \langle g \rangle_Y \]

  Also gilt:
  \[\abs{\langle g_\eps, \iW \rangle - \langle \langle g \rangle_Y, \iW
      \rangle} = 0, \quad \forall W\]
  Und mit der Dichtheit der Würfel in $L_2$ folgt die Aussage.

% ==================== 11.4 ====================
\item
  \[a_n = b_n \coloneqq \indn \sqrt{n} \in L^\infty(\Omega) \subset L^2(\Omega)\]
  \[\int_\Omega a_n f \dx \leq
    \underbrace{\sqrt{\int_0^{\frac{1}{n}} \sqrt{n}^2\dx}}_{\rightarrow 1}
    \underbrace{\sqrt{\int_0^{\frac{1}{n}} f^2 \dx}}_{\rightarrow 0} \rightarrow 0 \]

Für $f \equiv 1 \in L^2(0, 1)$ gilt aber:
\[\int_\Omega a_n b_n f \dx = \int_0^{\frac{1}{n}} n \dx = 1 \rightarrow \infty \neq 0 =
  a b\]


% ==================== 11.5 ====================
\item
  \begin{enumerate}
  \item
  \item
  \end{enumerate}


% ==================== 11.6 ====================
\item
  \begin{enumerate}
  \item
  \item
  \item
  \end{enumerate}

\end{enumerate}
\end{document}