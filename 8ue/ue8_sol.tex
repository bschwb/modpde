\documentclass[a4paper,11pt]{scrartcl}
\usepackage[utf8]{inputenc}
\usepackage[ngerman]{babel}

\usepackage[headsepline]{scrlayer-scrpage}
\ohead{Bernd Schwarzenbacher}
\ihead{UE8}

\usepackage{amsmath}
\usepackage{amssymb}
\usepackage{commath}
\usepackage{mathtools}
\usepackage[retainorgcmds]{IEEEtrantools}

\usepackage{siunitx}
\usepackage{enumitem}

\newcommand*{\eps}{\varepsilon}
\newcommand*{\sm}{\sum_{i=0}^\infty}
\newcommand*{\Ld}{\mathcal{O}}
\newcommand*{\dx}{\dif{}x}

\setlist[enumerate,2]{label=\textbf{\alph*)}}

\sisetup{per-mode=fraction}

\begin{document}


\begin{enumerate}[label*=\textbf{8.\arabic*.}]

% ==================== 8.1 ==================== 
\item \begin{enumerate}
  \item
    Für die ``äußere Entwicklung'' verwenden wir den Reihenansatz wie bei regulär gestörten Problemen:
    $y = \sm \eps^i y_i$, setzen in die Gleichung ein und machen einen
    Koeffizientenvergleich bezüglich $\eps$:
    \[ \sm \eps^{i+1} y''_i + 2 \sm \eps^{i} y'_i + \sm \eps^i y_i = e^x \]
    \[ 2y'_0 + y_0 = e^x, \quad y_0(1)=0 \]
    \[ 2y'_i + y_i = -y''_{i-1}, \quad y_i(1)=0, \quad \forall i \geq 1\]

  \item
    Wir führen die Grenzschichtvariable $\xi = \frac{x}{\eps}$ ein. (Warum nicht
    $\eps^\alpha$?)

    \[Y(\xi) = y(x) \Rightarrow \frac{1}{\eps} Y'(\xi) = y'(x),
      \frac{1}{\eps^2} Y''(\xi) = y''(x)\]
    \[\eps^{-1} Y'' + 2 \eps^{-1} Y' + Y = e^{\eps \xi} = \sm \frac{(\eps \xi)^i}{i!}\]
    \[Y(\xi) = \sm \eps^i Y_i\]
    \[\sm \eps^{i-1} Y''_i + 2 \sm \eps^{i-1} Y'_i + \sm \eps^i Y_i = \sm
      \frac{(\eps \xi)^i}{i!}\]
    \[ Y''_0 + 2 Y'_0 = 0, \quad Y_0(0) = 0\]
    \[ Y''_1 + 2 Y'_1 + Y_0 = 1, \quad Y_1(0) = 0\]
    \[ Y''_i + 2 Y'_i + Y_{i-1} = \frac{(\eps \xi)^i}{i!}, \quad Y_i(0) = 0\]

  \item
    Ansatz $y_0 = a e^x$:
    \[2ae^x + a e^x = e^x \Rightarrow a = \frac{1}{3} \Rightarrow y_{0p} =
      \frac{1}{3}e^x \]
    Charakteristisches Polynom für die homogene Lösung: 
    \[2 \lambda + 1 = 0 \Rightarrow \lambda=-\frac{1}{2}\]
    \[0 = y_0(1) = y_{0p}(1) + b y_{0h}(1) = \frac{1}{3} e + b e^{-\frac{1}{2}}
      \Rightarrow b = - \frac{1}{3} e^{\frac{3}{2}} \]
    \[y_0 = \frac{1}{3}e^x - \frac{1}{3} e^{\frac{3}{2} - \frac{1}{2}x}\]
    \[y''_0 = \frac{1}{3}e^x - \frac{1}{12} e^{\frac{3}{2} - \frac{1}{2}x}\]
    \[ 2 y'_1 + y_1 = -\frac{1}{3} e^x + \frac{1}{12}e^{\frac{3}{2} - \frac{1}{2}x}\]
    Ansatz:
    \[y_{1p} = c_1 e^x + c_2 x e^{-\frac{1}{2}x}\]
    \[2 c_1 e^x - c_2 x e^{-\frac{1}{2}x} + 2c_2 e^{-\frac{1}{2}x} + c_1 e^x +
      c_2 x e^{-\frac{1}{2} x}=  -\frac{1}{3} e^x + \frac{1}{12}e^{\frac{3}{2} - \frac{1}{2}x}\]
    \[3c_1 = -\frac{1}{3} \Rightarrow c_1 = -\frac{1}{9}\]
    \[2c_2 = \frac{1}{12} e^{\frac{3}{2}} \Rightarrow c_2 = \frac{1}{24}e^{\frac{3}{2}}\]
    \[0 = y_1(1) = y_{1p}(1) + b y_{1h}(1) = -\frac{1}{9}e + \frac{1}{24} e + b
      e^{-\frac{1}{2}} \]
    \[\Rightarrow b = \frac{5}{72} e^{\frac{3}{2}} \]
    \[y_1 = -\frac{1}{9}e^x +  (3x + 5) \frac{1}{72}e^{\frac{3}{2}-\frac{1}{2}x} \]

    Lösungen für $Y_0$:
    \[ Y_{0, a} = a\]
    Charakteristisches Polynom für $Y_0$:
    \[\lambda^2 + 2 \lambda = 0 \Rightarrow Y_{0, b} = e^{-2\xi}\]
    \[0 = Y_0(0) = a + b e^{-2 \cdot 0} \Rightarrow Y_0 = a (1 - e^{-2\xi}) \]
    \[Y''_1 + 2 Y'_1 = 1 + a (e^{-2 \xi} - 1)\]
    \[Y_1(\xi) = -\frac{1}{2} a e^{-2\xi} \xi - \frac{a\xi}{2} -\frac{1}{4} c_1
      e^{-2 \xi} + c_2 + \frac{\xi}{2} \]
    \[0= Y_1(0) = -\frac{c_1}{4} + c_2 \Rightarrow c_2 = \frac{1}{4} c_1\]
    Wolframalpha anscheinend:
    \[Y_1 = -e^{-2\xi} \left(\frac{1}{2}a \xi - c_2\right) + \frac{1-a}{2} \xi
      + c_2\]

    Matching:
    \[\lim_{x\rightarrow 0} y_0(x) = \lim_{\xi \rightarrow \infty} Y_0(x)
      \Rightarrow \frac{1}{3}( 1 -e^{\frac{3}{2}}) = a\]

    \[\zeta = \frac{x}{\eta(\eps)}\]
    \[y_0(x) + \eps y_1(x) = \frac{1}{3} e^{\eta\zeta} - \frac{1}{3}
      e^{\frac{3}{2} - \frac{1}{2}\eta\zeta} - \eps \frac{1}{3} e^{-\zeta
        \eta}+ \eps(3 \zeta \eta + 5) \frac{1}{72} e^{\frac{3}{2} - \frac{1}{2}
        \zeta \eta}\]
    \[ = \frac{1}{3} (1 + \eta \zeta) - \frac{1}{3} (e^{\frac{3}{2}}
      -\frac{1}{2} \eta \zeta e^{\frac{3}{2}}) - \eps \frac{1}{3} (1 - \eta
      \zeta) + \eps (3 \eta \zeta + 5) \frac{1}{72} (e^{\frac{3}{2}}
      -\frac{1}{2} \zeta \eta e^{\frac{3}{2}}) \]
    \[= \left(\frac{1}{3} -\frac{1}{3} e^{\frac{3}{2}} - \eps \frac{1}{3} + \eps
        \frac{5}{72} e^{\frac{3}{2}}\right)
    + \eta\left( \frac{1}{3} \zeta + \frac{1}{6} \zeta
        e^{\frac{3}{2}} + \eps \frac{1}{3} \zeta + \eps \frac{3}{72} \zeta
        e^{\frac{3}{2}} - \eps \frac{5}{72} \frac{1}{2}\zeta e^{\frac{3}{2}}
      \right) + \Ld(\eta^2) \]

    \[Y_0(\xi)+\eps Y_1(\xi) = a (1 - e^{-2 \frac{\zeta \eta}{\eps}}) - \eps e^{-2\h}  \]
    % Ansatz:
    % \[Y_{1p} = c_1 e^{-2\xi} + c_2 \xi e^{-2 \xi}\]
    % Einsetzen:
    % \[4 c_1 e^{-2\xi} + (-2 c_2 \xi e^{-2\xi} + c_2 e^{-2\xi})'- 4 c_1
    %   e^{-2\xi} - 4 c_2 \xi e^{-2 \xi} + 2 c_2 e^{-2\xi}} \]
    % \[=4 c_1 e^{-2\xi} +4 c_2 \xi e^{-2\xi} - 2 c_2 e^{-2\xi} -2 c_2 e^{-2\xi} - 4 c_1
    %   e^{-2\xi} - 4 c_2 \xi e^{-2 \xi} + 2 c_2 e^{-2\xi}} \]
    % \[= -2c_2 e^{-2 \xi} = 1 + a(e^{-2\xi} - 1)\]
    % \[\Rightarrow c_2 = \frac{1 + a(e^{-2\xi} - 1)}{-2 e^{-2\xi}} =
    %   -\frac{1}{2} (e^{2\xi} + a - a e^{2\xi})\]
    % \[0 = Y_1(0) = Y_{1p}(0) + b Y_{1h}(0) = c_1 + b d + b\]
      


\end{enumerate}

% ==================== 8.2 ==================== 
\item \begin{enumerate}
  \item
  \item
  \item
  \item
\end{enumerate}

% ==================== 8.3 ==================== 
\item \begin{enumerate}
  \item
  \item
\end{enumerate}

% ==================== 8.4 ==================== 
\item \begin{enumerate}
  \item
  \item
  \item
  \item
\end{enumerate}

% ==================== 8.5 ==================== 
\item

\end{enumerate}

\end{document}