\documentclass[a4paper,11pt]{scrartcl}
\usepackage[utf8]{inputenc}
\usepackage[english]{babel}

\usepackage[headsepline]{scrlayer-scrpage}
\ohead{Bernd Schwarzenbacher}
\ihead{UE3}

\usepackage{amsmath}
\usepackage{amssymb}
\usepackage{commath}
\usepackage{mathtools}
\usepackage[retainorgcmds]{IEEEtrantools}

\usepackage{enumitem}

\newcommand*{\R}{\mathbb{R}}
\newcommand*{\dx}{\dif{}x}
\newcommand*{\dt}{\dif{}t}
\newcommand*{\Ix}[1]{\int_{\R}{#1}\dx}
\newcommand*{\It}[1]{\int_0^\infty{#1}\dt}

\begin{document}

\begin{enumerate}[label*=\textbf{3.\arabic*.}]

\item
  Die schwache Formulierung lautet:

  \[ \Ix{\It{-f(u) \Phi_x - u \Phi_t}} - \Ix{u_0 \Phi_0} = \Ix{\It{g(u) \Phi}} \]

  Sei nun $u$ eine klassische Lösung mit Schock in $\psi(t)$.

  Wobei wie gewohnt $u_l(t) = \displaystyle{\lim_{x \nearrow \psi(t)} u(x, t)}$,
  $u_r(t) = \displaystyle{\lim_{x \searrow \psi(t)} u(x, t)}$.

  Führt man die Herleitung der RH-Bedingung wie im Skript durch, ergibt sich:
  \begin{IEEEeqnarray*}{rCl}
   \Ix{\It{g(u)\Phi}} & = & \Ix{\It{f(u)_x\Phi + u_t\Phi}} \\
    & & + [\psi'(t)(u_l(t) - u_r(t)) - f(u_l(t) + f(u_r(t))] \It{\Phi(\psi(t), t)} \\
  \end{IEEEeqnarray*}
Damit sich alles auf $0$ ergibt, muss also wieder gelten:
\[ \psi'(t) = \frac{f(u_l(t)) - f(u_r(t))}{u_l(t) - u_r(t)}\]

\item

\item
  \begin{enumerate}[label=\textbf{\alph*)}]
    \item Der Beweis sollte wieder so funktionieren wie die RH-Bedingung.
    \item Bsp. 1.11 durchführen mit $\eta, \psi$ allgemein?
    \item
  \end{enumerate}


\item
  \begin{enumerate}[label=\textbf{\alph*)}]
  \item
    \[ f(u) = \eta(u) = \frac{u^2}{2} \Rightarrow f'(u) = \eta'(u) = u\]
    \[\psi' = f' \eta' \Rightarrow \psi' = u^2 \Rightarrow \psi = \frac{u^3}{3} \]

  \end{enumerate}

\item

\end{enumerate}

\end{document}