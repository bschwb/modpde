\documentclass[a4paper,11pt]{scrartcl}
\usepackage[utf8]{inputenc}
\usepackage[english]{babel}

\usepackage[headsepline]{scrlayer-scrpage}
\ohead{Bernd Schwarzenbacher}
\ihead{UE3}

\usepackage{amsmath}
\usepackage{amssymb}
\usepackage{commath}
\usepackage{mathtools}
\usepackage[retainorgcmds]{IEEEtrantools}

\usepackage{enumitem}

\newcommand*{\R}{\mathbb{R}}
\newcommand*{\F}{\mathbf{F}}
\newcommand*{\dx}{\dif{}x}
\newcommand*{\dt}{\dif{}t}
\newcommand*{\du}{\dif{}u}
\newcommand*{\Ix}[1]{\int_{\R}{#1}\dx}
\newcommand*{\It}[1]{\int_0^\infty{#1}\dt}
\newcommand*{\IR}{\int_{\R^2}}

\begin{document}

\begin{enumerate}[label*=\textbf{3.\arabic*.}]

\item
  Die schwache Formulierung lautet:

  \[ \Ix{\It{-f(u) \Phi_x - u \Phi_t}} - \Ix{u_0 \Phi_0} = \Ix{\It{g(u) \Phi}} \]

%   Sei nun $u$ eine klassische Lösung mit Schock in $\psi(t)$.

%   Wobei wie gewohnt $u_l(t) = \displaystyle{\lim_{x \nearrow \psi(t)} u(x, t)}$,
%   $u_r(t) = \displaystyle{\lim_{x \searrow \psi(t)} u(x, t)}$.

%   Führt man die Herleitung der RH-Bedingung wie im Skript durch, ergibt sich:
%   \begin{IEEEeqnarray*}{rCl}
%    \Ix{\It{g(u)\Phi}} & = & \Ix{\It{f(u)_x\Phi + u_t\Phi}} \\
%     & & + [\psi'(t)(u_l(t) - u_r(t)) - f(u_l(t) + f(u_r(t))] \It{\Phi(\psi(t), t)} \\
%   \end{IEEEeqnarray*}
% Damit sich alles auf $0$ ergibt, muss also wieder gelten:
% \[ \psi'(t) = \frac{f(u_l(t)) - f(u_r(t))}{u_l(t) - u_r(t)}\]

\item

\item
  \begin{enumerate}[label=\textbf{\alph*)}]
  \item
    Damit $u$ eine Entropielösung ist, muss sie zunächst eine schwache Lösung
    sein, also die RH-Bedingung erfüllen.
    Zusätzlich:
    \begin{IEEEeqnarray*}{Cl}
      &\It{\Ix{\eta(u)\Phi_t + \psi(u)\Phi_x}} \\
      =&\It{\left[\partial_t\int_{-\infty}^{st} \eta(u) \Psi \dx - s \eta(u_l) \Phi(st, t)
        + \partial_t \int_{st}^\infty \eta(u) \Phi + s \eta(u_r) \Phi(st, t)\right]}\\
      & + \It{\psi(u_l)\Phi(st,t) - \psi(u_r)\Phi(st,t)} \\
      =& - \Ix{\eta(u_0) \Phi(x,0)} - \left(s (\eta(u_l) - \eta(u_r)) +
        \psi(u_l) - \psi(u_r)\right) \It{\underbrace{\Phi(st, t)}_{\geq 0}} \\
    \end{IEEEeqnarray*}
    $u$ erfüllt also Entropiebedingung wenn der zweite Term negativ ist, was
    äquivalent ist zu:
    \[s(\eta(u_r) - \eta(u_l)) \geq \psi(u_r) - \psi(u_l)\]
    für alle Entropie-Entropiefluss-Paare.

    \item 
      \[f(u) = \frac{u^2}{2} \Rightarrow f'(u) = u \Rightarrow \psi' = f' \eta'
        = u \eta'\]

      \[ s = \frac{u_l + u_r}{2} \Rightarrow s - u_r = \frac{u_l - u_r}{2}, \;
        u_l - s = \frac{u_l - u_r}{2} \]

      \begin{IEEEeqnarray*}{Cl}
        &s(\eta(u_r) - \eta(u_l) + \psi(u_l) - \psi(u_r)) = \int_{u_l}^{u_r} s
        \eta'(u) \dif{}u + \int_{u_r}^{u_l} \psi'(u) \du \\
        =& \int_{u_r}^s (u-s) \eta'(u) \du + \int_s^{u_l} (u-s)\eta'(u) \du \\
        =& \int_{0}^{s-u_r} -u \eta'(s-u) \du + \int_0^{u_l-s} u\eta'(s+u) \du \\
        =& \int_0^{\frac{u_l-u_r}{2}} u (\eta'(s-u) + \eta'(s+u)) \du \geq 0
      \end{IEEEeqnarray*}
      Der letzte Schritt folgt weil $u>0$ in den Integrationsgrenzen und weil $\eta$ konvex ist.

    \item
      Schwache Lösung wissen wir aus der Vorlesung mit $s = \frac{u_l+u_r}{2}$.
      Setze in die Ungleichung von \textbf{a)} ein: $\eta(u) = u^2,
      \psi' = f' \eta' = 2 u^2 \Rightarrow \psi(u) = \frac{2}{3} u^3$
      \begin{IEEEeqnarray*}{rCl}
      s (u_r^2 - u_l^2) = \frac{u_r^3 + u_lu_r^2 - u_ru_l^2 - u_l^3}{2} &\geq&
      \frac{2(u_r^3 - u_l^3)}{3} \\
      3u_r^3 + 3u_lu_r^2 - 3u_ru_l^2 - 3u_l^3 &\geq& 4(u_r^3 - u_l^3) \\
      -u_r^3 + 3u_lu_r^2 - 3u_ru_l^2 + u_l^3 &\geq& 0 \\
      (u_l - u_r)^3\geq 0 &\iff& u_l \geq u_r \\
      \end{IEEEeqnarray}


  \end{enumerate}


\item
  \begin{enumerate}[label=\textbf{\alph*)}]
  \item
    \[ f(u) = \eta(u) = \frac{u^2}{2} \Rightarrow f'(u) = \eta'(u) = u\]
    \[\psi' = f' \eta' \Rightarrow \psi' = u^2 \Rightarrow \psi = \frac{u^3}{3} \]

  \item
    Start bei der Entropiebedingung (1.20):
    \[ \It{\Ix{[\eta(u)\Phi_t + \psi(u)\Phi_x]}} \geq - \Ix{\eta(u_0(x))\Phi(x, 0)}\]
    mit $\eta, \psi$ aus \textbf{a)}.

    Analog zu den Kruzkov-notes nehmen wir:
    \[ \F = \begin{pmatrix}\psi(u)\\ \eta(u)\end{pmatrix} =
      \begin{pmatrix}
        \frac{u^3}{3}\\\frac{u^2}{2}
      \end{pmatrix}
    \]
    und als Testfunktion die Indikatorfunktion $\chi_K$,
    wobei $K$ das Trapez mit den Eckpunkten
    $(-R, t_1), (R, t_1), (R-M(t_2-t_1), t_2), (-R+M(t_2-t_1), t_2)$ ist.
    \[\Gamma_L = \{(x,t) | t_1 < t < t_2, \, \abs{x} = R - M(t-t_1)\}\]

    \begin{IEEEeqnarray*}{rCl}
      0 &\leq& \IR \F \cdot \nabla_{x,t} \chi_K
      = - \IR \nabla_{x,t}\cdot F\chi_K
      = \int_K \nabla_{x,t} \cdot \F
      = - \int_{\partial K} \F \cdot
      \mathbf{n}_K \\
      &=& - \left[\int_{\abs{x} \leq R - M(t_2 - t_1)} \frac{u^2(x, t_2)}{2} \dx -
        \int_{\abs{x}\leq R} \frac{u^2(x, t_1)}{2} \dx
        + \int_{\Gamma_L} \frac{u^3}{3} \frac{x}{\abs{x}} + M \frac{u^2}{2}
      \right]
    \end{IEEEeqnarray*}
    Weil $u \in L^\infty$ wählen wir
    \[ M \coloneqq \sup \frac{2\norm{u}_\infty}{3}\]
    und so wird das Integral über $\Gamma_L$ sicher positiv.
    Mit $R \rightarrow \infty$ und $t_1 \rightarrow 0$ folgt dann die Behauptung.

    Haben angenommen, dass die rechte Seite der Entropiebedingung immer $0$ ist.
    Das ergibt sich aus der speziellen Wahl unserer Testfunktionen, die sind für
    $t_1 > 0$ immer $0$ in $t=0$.
    und mit richtiger Reihenfolge der Grenzwertbildung sollte das passen.
  \end{enumerate}

\item
  \begin{enumerate}[label=\arabic*.]
  \item
    Gilt eh für alle schwachen Lösungen.
  \item
    \[ \Phi(u) = \abs{u} + u \Rightarrow \Phi'(u) = \text{sgn}(u) + 1\]
    \[ \Psi' = \Phi' f' = f' (\text{sgn}(u) + 1) \Rightarrow \Psi = f(u)
      (\text{sgn}(u) + 1)\]
  \item
  \end{enumerate}

\end{enumerate}

\end{document}