\documentclass[a4paper,11pt]{scrartcl}
\usepackage[utf8]{inputenc}
\usepackage[english]{babel}

\usepackage[headsepline]{scrlayer-scrpage}
\ohead{Bernd Schwarzenbacher}
\ihead{UE4}

\usepackage{amsmath}
\usepackage{amssymb}
\usepackage{commath}
\usepackage{mathtools}
\usepackage[retainorgcmds]{IEEEtrantools}

\usepackage{url}
\usepackage{enumitem}

\DeclareMathOperator{\spur}{spur}

\newcommand*{\Landau}{\mathcal{O}}
\newcommand*{\eps}{\varepsilon}
\newcommand*{\dx}{\dif{}x}
\newcommand*{\ds}{\dif{}s}
\newcommand*{\dt}{\dif{}t}

\begin{document}

\begin{enumerate}[label*=\textbf{4.\arabic*.}]
  \item
    \[ f(u) = \frac{u^2}{u^2+(1-u)^2} \]
    \begin{IEEEeqnarray*}{rCl}
    f'(u) &=& \frac{2u (u^2 + (1-u)^2) - u^2 (2 u - 2(1-u))}{(u^2 + (1-u)^2)^2} \\
    &=& \frac{2u^3 + 2u - 4u^2 + 2 u^3 - 2 u^3 + 2 u^2 - 2u^3}{(u^2 +
      (1-u)^2)^2} \\
    &=& \frac{2u(1-u)}{(u^2 + (1-u)^2)^2}
    \end{IEEEeqnarray*}
    \[ (f')^{-1}(\xi) = \frac{1}{2}
      \left(1 - \sqrt{
          \frac{1}{\xi}
          \left(
            \sqrt{4\xi + 1} - 1
          \right)
          - 1} \right) \]


Wir verwenden den Hinweis, dass die Lösung ein Verdünnungsfächer gefolgt von
einem Schock der Form $x=st$ ist:

  \[ u(x, t) =
      \begin{cases}
        u_1 & x < f'(u_1)t\\
        (f')^{-1}(\frac{x}{t}) & f'(u_1) t \leq x \leq s t\\
        u_2 & st < x \\
      \end{cases}
  \]

Damit die Anfangsbedingungen erfüllt sind, muss also $u_1 = 0$ und $u_2 = 1$ sein.

  \[ u(x, t) =
      \begin{cases}
        0 & x < 0 \\
        (f')^{-1}(\frac{x}{t}) & 0 \leq x \leq s t\\
        1 & st < x \\
      \end{cases}
  \]

  \item
    \[v(x_1, x_2) = \begin{pmatrix}x_2\\x_1\end{pmatrix}\]

  Stromlinien sind die Integralkurven vom Geschwindigkeitsfeld $v$, also die
  Lösungen von $\frac{\dx}{\ds} = v(x;t)$ für eine feste Zeit $t$. Das gegeben
  Geschwindigkeitsfeld ist stationär, daher sind Bahnlinien gegeben durch die Stromlinien.

  Die Lösungen dieser DGL sind von der Form:
  \[\begin{pmatrix}x_1\\x_2\end{pmatrix}(s) =
    \begin{pmatrix}\sinh(s)+c_1\\\cosh(s)+c_2\end{pmatrix}\]

  \item
    \[\det(B + \eps C B) - \det(B) = \left[\det(I + \eps C) - 1\right] \det(B) \]
    Mit der Leibniz Formel für Determinanten erhalten wir:
    \begin{IEEEeqnarray*}{Cl}
    =&  \left[\prod^n_{i=1} (1 + \eps c_{ii}) + \Landau(\eps^2) -
      1\right] \det(B) = \left[1 + \eps \sum_{i=1}^n c_{ii} + \Landau(\eps^2) -
      1\right] \det(B) \\
    =&\eps \det(B) \spur(C) + \Landau(\eps^2)
    \end{IEEEeqnarray}

    Gleicher Beweis, wenn $B$ und $C$ vertauscht sind, deshalb
    \begin{IEEEeqnarray*}{rCl}
      \frac{\dif{}}{\dt}A(t) &=& \lim_{\eps \rightarrow 0}\frac{\det A(t + \eps) -
        \det A(t)}{\eps} \\
      &=&\lim_{\eps \rightarrow 0}\frac{\det (A(t) + \eps A(t) A^{-1}(t)A'(t) +
        \Landau(\eps^2)) - \det A(t)}{\eps} \\
    \end{IEEEeqnarray*}
    mit wähle $C = A^{-1} A'$.

  \item

\end{enumerate}

\end{document}