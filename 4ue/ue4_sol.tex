\documentclass[a4paper,11pt]{scrartcl}
\usepackage[utf8]{inputenc}
\usepackage[english]{babel}

\usepackage[headsepline]{scrlayer-scrpage}
\ohead{Bernd Schwarzenbacher}
\ihead{UE4}

\usepackage{amsmath}
\usepackage{amssymb}
\usepackage{commath}
\usepackage{mathtools}
\usepackage[retainorgcmds]{IEEEtrantools}

\usepackage{enumitem}

\newcommand*{\dx}{\dif{}x}
\newcommand*{\ds}{\dif{}s}

\begin{document}

\begin{enumerate}[label*=\textbf{4.\arabic*.}]
  \item
    \[ f(u) = \frac{u^2}{u^2+(1-u)^2} \]
    \begin{IEEEeqnarray*}{rCl}
    f'(u) &=& \frac{2u (u^2 + (1-u)^2) - u^2 (2 u - 2(1-u))}{(u^2 + (1-u)^2)^2} \\
    &=& \frac{2u^3 + 2u - 4u^2 + 2 u^3 - 2 u^3 + 2 u^2 - 2u^3}{(u^2 +
      (1-u)^2)^2} \\
    &=& \frac{-4u^2}{(u^2 + (1-u)^2)^2}
    \end{IEEEeqnarray*}
    \[ (f')^{-1}(\xi) = \frac{1}{2}\left( 1 - \sqrt{\frac{1}{\xi}\left( \sqrt{4\xi + 1}
          - 1} - 1\right)} \right) \]
    \[ u =
      \begin{cases}
        u_1 & x < f'(u_1)t\\
        (f')^{-1}(\frac{x}{t}) & f'(u_1) t \leq x \leq f'(u_2) t\\
        u_2 & f'(u_2) t < x < st  \\
        u_3 & st < x \\
      \end{cases}
\]

  \item
    \[v(x_1, x_2) = \begin{pmatrix}x_2\\x_1\end{pmatrix}\]

  Stromlinien sind die Integralkurven vom Geschwindigkeitsfeld $v$, also die
  Lösungen von $\frac{\dx}{\ds} = v(x;t)$ für eine feste Zeit $t$. Das gegeben
  Geschwindigkeitsfeld ist stationär, daher sind Bahnlinien gegeben durch die Stromlinien.

  Die Lösungen dieser DGL sind von der Form:
  \[\begin{pmatrix}x_1\\x_2\end{pmatrix}(s) =
    \begin{pmatrix}\sinh(s)+c_1\\\cosh(s)+c_2\end{pmatrix}\]

  \item
  \item
\end{enumerate}

\end{document}