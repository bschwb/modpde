\documentclass[a4paper,11pt]{scrartcl}
\usepackage[utf8]{inputenc}
\usepackage[english]{babel}

\usepackage[headsepline]{scrlayer-scrpage}
\ohead{Bernd Schwarzenbacher}
\ihead{UE6}

\usepackage{amsmath}
\usepackage{amssymb}
\usepackage{commath}
\usepackage{mathtools}
\usepackage[retainorgcmds]{IEEEtrantools}

\DeclareMathOperator*{\divop}{div}

\newcommand*{\dx}{\dif{}x}
\newcommand*{\ds}{\dif{}s}
\newcommand*{\dr}{\dif{}r}

\usepackage{enumitem}

\begin{document}

\begin{enumerate}[label*=\textbf{6.\arabic*.}]
  \item
  \begin{enumerate}[label=\textbf{\alph*)}]
    \item
      Dass $\divop{v} = 0$ erfüllt ist, folgt aus der Annahme der Gestalt von $v$.
      Weiters gilt
      \[(v \cdot \nabla) v =
        \begin{pmatrix}
          v_1 \partial_1 v_1 + v_2 \partial_2 v_1 + v_3 \partial_3 v_1\\
          v_1 \partial_1 v_2 + v_2 \partial_2 v_2 + v_3 \partial_3 v_2\\
          v_1 \partial_1 v_3 + v_2 \partial_2 v_3 + v_3 \partial_3 v_3
        \end{pmatrix}
        =
        \begin{pmatrix}0\\0\\0\end{pmatrix}
      \]

      \[\partial_{22} v_1
      = \partial_2\left(v_1'\frac{x_2}{r}\right)
      = v_1'' \frac{x^2_2}{r^2} + v_1' \left( \frac{1}{r} - \frac{x_2^2}{r^3} \right)
    \Rightarrow \Delta v_1 =
    v_1'' + v_1'\left( \frac{2}{r} - \frac{1}{r}\right) = v_1'' + \frac{1}{r} v_1' \]

    Also
    \[ \nabla p =
      \begin{pmatrix}
        \partial_1 p\\\partial_2 p \\\partial_3 p
      \end{pmatrix} =
      \eta \Delta v =
      \begin{pmatrix}\eta (v_1'' + \frac{1}{r} v_1')\\0\\0\end{pmatrix} \]

    Daher $p = p(x_1)$ und die linke Seite hängt nur von $x_1$ ab, während die
    rechte Seite nur von $x_2$ und $x_3$ abhängt. Also müssen beide Seiten
    konstant sein.

    \[p' = c \Rightarrow p = c x_1 + d\]
    \[p_1 = p(0) = d, \quad p_2 = p(L) = c L + p_1 \Rightarrow c = \frac{p_2 - p_1}{L}\]
    Mit Ansatz $v_1 = a r^2 + b$ gilt
    \[\frac{c}{\eta} = v_1'' + \frac{1}{r}v_1' = 2 a + 2 a = 4 a \Rightarrow a
      = \frac{c}{4\eta} = \frac{p_2 - p_1}{4\eta L}\]
    Weiters gilt:
    \[0 = v(R) \Rightarrow 0 = v_1(R) = a R^2 + b \Rightarrow b = - a R^2\]
    \[ v_1(r) = a (r^2 - R^2) = \frac{p_1-p_2}{4\eta L}(R^2 - r^2)\]
    
    \item
      \[\int_\Omega v_1 \dx = 2\pi \int_0^R v_1(r) r \dr =
        \frac{\pi(p_1-p_2)}{2\eta L}\left[\frac{R^4}{2} - \frac{r^4}{4}
        \right]_{r=0}^R
        = \frac{\pi(p_1-p_2)}{2\eta L} \frac{R^4}{4}
      \]

  \end{enumerate}

  \item
  \begin{enumerate}[label=\textbf{\alph*)}]
    \item
      \[\nabla \times \nabla \phi =
        \begin{pmatrix}
          \partial_2 \partial_3 \phi - \partial_3 \partial_2 \phi\\\dots\\\dots
        \end{pmatrix} = 0
\]
    \item
      \[0 = \nabla \cdot v = \nabla \cdot \nabla \phi = \Delta \phi\]

    \item
      Wir starten bei der Impulserhaltung:
      \[\partial_t (\rho v) + v \divop(\rho v) + \rho (v \cdot \nabla) v +
        \nabla p = \rho f\]
      \[\Rightarrow \nabla p = - \partial_t (\rho v) - v \divop(\rho v) - \rho (v \cdot \nabla) v \]
      \[\nabla p = - \partial_t \rho v - \rho \partial_t v - v \divop(\rho v) - \rho (v \cdot \nabla) v \]

      Aus der Massenerhaltung $\rho_t + \divop(\rho v) = 0$ folgt:
      \[\nabla p = -\rho \partial_t v - \rho (v \cdot \nabla) v \]
      \[\Rightarrow \int_C \frac{1}{\rho} \nabla p \cdot \ds
        = \int_C -\partial_t v  - (v \cdot \nabla) v \cdot \ds
        = \int_C -\partial_t \nabla \varphi  - (\nabla \varphi \cdot \nabla) \nabla \varphi \cdot \ds
        \]
      \[
      = \int_C -\partial_t \nabla \varphi  - \nabla \left( \frac{1}{2}
        \norm{\nabla \varphi}^2 \right)\cdot \ds
      = \left[-\partial_t \varphi - \frac{1}{2}\norm{v}^2  \right]_{x_0}^{x_1}
      \]

    \item
      Inkompressible Navier-Stokes Gleichungen für homogenes Fluid:
      \begin{IEEEeqnarray*}{rCl}
        \rho [v_t + (v \cdot \nabla) v ] + \nabla p &=& \Delta v + \rho f \\
        \divop v &=& 0
      \end{IEEEeqnarray*}

      Annahmen: stationär,  $f=0$ und Potentialströmung
      \[ \Delta v_i = \Delta \partial_i \varphi = \partial_i \Delta \varphi = 0\]
      \[\Rightarrow \nabla p = - \rho (v \cdot \nabla) v\]
      \[\Rightarrow p = - \rho \frac{1}{2} \norm{v}^2 + c\]

  \end{enumerate}

  \item
    Transformation der NS-Gleichungen auf Zylinderkoordinaten:

    \[(u1, u2, u3) = u_r (\cos \theta, \sin \theta, 0)  + u_\theta (-sin \theta,
      cos \theta, 0) + u_z (0, 0, 1)\]
    \[ \rho \left(\frac{D u_r}{D t} - \frac{u_\theta^2}{r}\right) = \rho f_r - \frac{\partial
        p}{\partial r} + \mu \left[ \Delta u_r - \frac{u_r}{r^2} -
        \frac{2}{r^2}\frac{\partial u_\theta}{\partial \theta} \right]\]
    \[ \rho \left(\frac{D u_\theta}{D t} + \frac{u_r u_\theta}{r}\right) = \rho f_\theta -
      \frac{1}{r} \frac{\partial p}{\partial \theta} + \mu \left(\Delta u_0 +
      \frac{2}{r^2} \frac{\partial u_r}{\partial \theta} - \frac{u_0}{r^2}\right)\]
    \[\rho \frac{D u_z}{D t} = \rho f_z - \frac{\partial p}{\partial z} + \mu
      \Delta u_z\]

    \[\Delta \varphi = \frac{1}{r} \frac{\partial}{\partial r} \left(  r
        \frac{\partial\varphi}{\partial r} + \frac{1}{r^2} \frac{\partial^2
          \varphi}{\partial \theta^2} + \frac{\partial^2 \varphi}{\partial z^2} \right)\]
    \[ \frac{D\varphi}{D t} = \frac{\partial\varphi}{\partial t}+ u_r \frac{\partial
        \varphi}{\partial r} + \frac{u_\theta}{r} \frac{\partial \varphi}{\partial
        \theta} + u_z \frac{\partial \varphi}{\partial z}\]

\end{enumerate}

\end{document}