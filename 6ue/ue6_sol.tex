\documentclass[a4paper,11pt]{scrartcl}
\usepackage[utf8]{inputenc}
\usepackage[english]{babel}

\usepackage[headsepline]{scrlayer-scrpage}
\ohead{Bernd Schwarzenbacher}
\ihead{UE6}

\usepackage{amsmath}
\usepackage{amssymb}
\usepackage{commath}
\usepackage{mathtools}
\usepackage[retainorgcmds]{IEEEtrantools}

\DeclareMathOperator*{\divop}{div}

\newcommand*{\dx}{\dif{}x}
\newcommand*{\dr}{\dif{}r}

\usepackage{enumitem}

\begin{document}

\begin{enumerate}[label*=\textbf{6.\arabic*.}]
  \item
  \begin{enumerate}[label=\textbf{\alph*)}]
    \item
      Dass $\divop{v} = 0$ erfüllt ist, folgt aus der Annahme der Gestalt von $v$.
      Weiters gilt
      \[(v \cdot \nabla) v =
        \begin{pmatrix}
          v_1 \partial_1 v_1 + v_2 \partial_2 v_1 + v_3 \partial_3 v_1\\
          v_1 \partial_1 v_2 + v_2 \partial_2 v_2 + v_3 \partial_3 v_2\\
          v_1 \partial_1 v_3 + v_2 \partial_2 v_3 + v_3 \partial_3 v_3
        \end{pmatrix}
        =
        \begin{pmatrix}0\\0\\0\end{pmatrix}
      \]

      \[\partial_{22} v_1
      = \partial_2\left(v_1'\frac{x_2}{r}\right)
      = v_1'' \frac{x^2_2}{r^2} + v_1' \left( \frac{1}{r} - \frac{x_2^2}{r^3} \right)
    \Rightarrow \Delta v_1 =
    v_1'' + v_1'\left( \frac{2}{r} - \frac{1}{r}\right) = v_1'' + \frac{1}{r} v_1' \]

    Also
    \[ \nabla p =
      \begin{pmatrix}
        \partial_1 p\\\partial_2 p \\\partial_3 p
      \end{pmatrix} =
      \eta \Delta v =
      \begin{pmatrix}\eta (v_1'' + \frac{1}{r} v_1')\\0\\0\end{pmatrix} \]

    Daher $p = p(x_1)$ und die linke Seite hängt nur von $x_1$ ab, während die
    rechte Seite nur von $x_2$ und $x_3$ abhängt. Also müssen beide Seiten
    konstant sein.

    \[p' = c \Rightarrow p = c x_1 + d\]
    \[p_1 = p(0) = d, \quad p_2 = p(L) = c L + p_1 \Rightarrow c = \frac{p_2 - p_1}{L}\]
    Mit Ansatz $v_1 = a r^2 + b$ gilt
    \[\frac{c}{\eta} = v_1'' + \frac{1}{r}v_1' = 2 a + 2 a = 4 a \Rightarrow a
      = \frac{c}{4\eta} = \frac{p_2 - p_1}{4\eta L}\]
    Weiters gilt:
    \[0 = v(R) \Rightarrow 0 = v_1(R) = a R^2 + b \Rightarrow b = - a R^2\]
    \[ v_1(r) = a (r^2 - R^2) = \frac{p_1-p_2}{4\eta L}(R^2 - r^2)\]
    
    \item
      \[\int_\Omega v_1 \dx = 2\pi \int_0^R v_1(r) r \dr =
        \frac{\pi(p_1-p_2)}{2\eta L}\left[\frac{R^4}{2} - \frac{r^4}{4}
        \right]_{r=0}^R
        = \frac{\pi(p_1-p_2)}{2\eta L} \frac{R^4}{4}
      \]

  \end{enumerate}

  \item
  \begin{enumerate}[label=\textbf{\alph*)}]
    \item
    \item
    \item
    \item
  \end{enumerate}
  \item

\end{enumerate}

\end{document}