\documentclass[a4paper,11pt]{scrartcl}
\usepackage[ngerman]{babel}
\usepackage[utf8]{inputenc}

\usepackage{amssymb}
\usepackage{amsmath}
\usepackage{commath}
\usepackage[retainorgcmds]{IEEEtrantools}

\usepackage{enumitem}

\newcommand*{\eps}{\varepsilon}
\newcommand*{\Ld}{\mathcal{O}}
\newcommand*{\Wh}{\widehat{W}}

\setlist[enumerate,2]{label=\textbf{\alph*)}}

\begin{document}
\begin{enumerate}[label*=\textbf{10.\arabic*.}]

% ==================== 10.1 ====================
\item
  Wir werden folgende Eigenschaft des Frobenius-Produkts brauchen (wir verwenden
  hier die Einstein'sche Summenkonvention):
  \[A : (B C) = A_{ij} (B C)_{ij} = A_{ij} B_{ik} C_{kj}
    = A_{ij} B^\top_{ki} C_{kj} = B^\top_{ki} A_{ij} C_{kj} = (B^\top A)_{kj} C_{kj}
    = (B^\top A) : C\]
  Außerdem gilt $A : B = B : A = A^\top : B^\top $ und aus der Vorlesung wissen
  wir, dass $\frac{\partial W(C)}{\partial C}$ symmetrisch ist.

  \begin{IEEEeqnarray*}{rCl}
    \Wh(F + \Delta) &=& W((F + \Delta)^\top(F+\Delta)) = W(F^\top F + F^\top
    \Delta + F \Delta^\top + \Delta^T \Delta) \\
    &=& W(F^\top F) + \frac{\partial W(F^\top F)}{\partial C}
    : (F^\top \Delta + \Delta^\top F + \Delta^T \Delta)
    + \Ld (\norm{\Delta}^2)\\
    &=& W(F^\top F) + F \frac{\partial W(F^\top F)}{\partial C} : \Delta +
    \frac{\partial W(F^\top F)}{\partial C} : (\Delta^T F) + \Ld (\norm{\Delta}^2) \\
    &=& W(F^\top F) + 2 F \frac{\partial W(F^\top F)}{\partial C} : \Delta
+ \Ld(\norm{\Delta}^2) \\
  \end{IEEEeqnarray*}
  \[\Rightarrow \frac{\partial \Wh(F)}{\partial F} = 2 F \frac{\partial W(F^\top F)}{\partial C}\]

% ==================== 10.2 ====================
\item

% ==================== 10.3 ====================
\item
  Man wähle die zwei folgenden Matrizen mit positiver Determinante:
  \[
    A = \begin{pmatrix} 1 & 0 \\ 0 & 1\end{pmatrix}, \quad
    B = \begin{pmatrix} -1 & 0 \\ 0 & -1\end{pmatrix}
  \]
  Die Konvexkombination $(1 - t) A + t B$ hat für $t = \frac{1}{2}$ Determinante
  gleich 0.
  Die Aussage lässt sich für alle $d \geq 2$ verallgemeinern, indem man die
  ersten zwei Diagonaleinträge wie oben wählt und die restliche Diagonale mit Einsern füllt.

% ==================== 10.4 ====================
\item
  \begin{enumerate}
  \item
  \item
  \end{enumerate}

% ==================== 10.5 ====================
\item

% ==================== 10.6 ====================
\item

\end{enumerate}
\end{document}