\documentclass[a4paper,11pt]{scrartcl}
\usepackage[ngerman]{babel}
\usepackage[utf8]{inputenc}

\usepackage{amssymb}
\usepackage{amsmath}
\usepackage{commath}
\usepackage[retainorgcmds]{IEEEtrantools}

\usepackage{enumitem}

\newcommand*{\eps}{\varepsilon}

\setlist[enumerate,2]{label=\textbf{\alph*)}}

\begin{document}
\begin{enumerate}[label*=\textbf{10.\arabic*.}]

% ==================== 10.1 ====================
\item

% ==================== 10.2 ====================
\item

% ==================== 10.3 ====================
\item
  Man wähle die zwei folgenden Matrizen mit positiver Determinante:
  \[
    A = \begin{pmatrix} 1 & 0 \\ 0 & 1\end{pmatrix}, \quad
    B = \begin{pmatrix} -1 & 0 \\ 0 & -1\end{pmatrix}
  \]
  Die Konvexkombination $(1 - t) A + t B$ hat für $t = \frac{1}{2}$ Determinante
  gleich 0.
  Die Aussage lässt sich für alle $d \geq 2$ verallgemeinern, indem man die
  ersten zwei Diagonaleinträge wie oben wählt und die restliche Diagonale mit Einsern füllt.

% ==================== 10.4 ====================
\item
  \begin{enumerate}
  \item
  \item
  \end{enumerate}

% ==================== 10.5 ====================
\item

% ==================== 10.6 ====================
\item

\end{enumerate}
\end{document}