\documentclass[a4paper,11pt]{scrartcl}
\usepackage[ngerman]{babel}
\usepackage[utf8]{inputenc}

\usepackage[headsepline]{scrlayer-scrpage}
\ohead{Bernd Schwarzenbacher}
\ihead{UE10}

\usepackage{amssymb}
\usepackage{amsmath}
\usepackage{commath}
\usepackage[retainorgcmds]{IEEEtrantools}

\usepackage{enumitem}

\newcommand*{\eps}{\varepsilon}
\newcommand*{\Ld}{\mathcal{O}}
\newcommand*{\Wh}{\widehat{W}}
\newcommand*{\Wt}{\tilde{W}}
\newcommand*{\dx}{\dif{}x}
\newcommand*{\ds}{\dif{}s_x}
\DeclareMathOperator*{\divop}{div}
\DeclareMathOperator*{\tr}{tr}


\setlist[enumerate,2]{label=\textbf{\alph*)}}

\begin{document}
\begin{enumerate}[label*=\textbf{10.\arabic*.}]

% ==================== 10.1 ====================
\item
  Wir werden folgende Eigenschaft des Frobenius-Produkts brauchen (wir verwenden
  hier die Einstein'sche Summenkonvention):
  \[A : (B C) = A_{ij} (B C)_{ij} = A_{ij} B_{ik} C_{kj}
    = A_{ij} B^\top_{ki} C_{kj} = B^\top_{ki} A_{ij} C_{kj} = (B^\top A)_{kj} C_{kj}
    = (B^\top A) : C\]
  Außerdem gilt $A : B = B : A = A^\top : B^\top $ und aus der Vorlesung wissen
  wir, dass $\frac{\partial W(C)}{\partial C}$ symmetrisch ist.

  \begin{IEEEeqnarray*}{rCl}
    \Wh(F + \Delta) &=& W((F + \Delta)^\top(F+\Delta)) = W(F^\top F + F^\top
    \Delta + F \Delta^\top + \Delta^T \Delta) \\
    &=& W(F^\top F) + \frac{\partial W(F^\top F)}{\partial C}
    : (F^\top \Delta + \Delta^\top F + \Delta^T \Delta)
    + \Ld (\norm{\Delta}^2)\\
    &=& W(F^\top F) + F \frac{\partial W(F^\top F)}{\partial C} : \Delta +
    \frac{\partial W(F^\top F)}{\partial C} : (\Delta^T F) + \Ld (\norm{\Delta}^2) \\
    &=& W(F^\top F) + 2 F \frac{\partial W(F^\top F)}{\partial C} : \Delta
+ \Ld(\norm{\Delta}^2) \\
  \end{IEEEeqnarray*}
  \[\Rightarrow \frac{\partial \Wh(F)}{\partial F} = 2 F \frac{\partial W(F^\top F)}{\partial C}\]

% ==================== 10.2 ====================
\item
  Mit $E = \frac{1}{2}(C - I)$ erhalten wir für $\Wt(E): \Wt(0) = 0$.

  Zunächst wird die Hilfsaussage laut Angabe bewiesen:
  \begin{IEEEeqnarray*}{rCl}
    \det(C) &=&
    \det(I + 2 E) =
    \det\begin{pmatrix}
     1 + 2e_{11} &2e_{12} &2e_{13} \\
     2e_{12} &1+2e_{22} &2e_{23} \\
     2e_{13} &2e_{23} &1+2e_{33}
    \end{pmatrix} \\
    &=& (1+2e_{11}) (1+2e_{22}) (1+2e_{33}) +
    16 e_{12} e_{23} e_{13} \\
    &&- (1+2e_{11}) 4 e_{23}^2 - (1+2e_{22}) 4 e_{13}^2
    - (1+2e_{33}) 4 e_{12}^2 \\
    &=& 1 + 2 \tr E + 2 (\tr E)^2 - 2 E : E + 8 \det(E)
  \end{IEEEeqnarray*}
  Wegen der Symmetrie von $E$ gilt außerdem: $E : E = \tr E^2$.
  Mit der Taylorentwicklung
  \[(1 + x)^{-\frac{\beta}{2}} = 1 - \frac{\beta}{2} x +
    \frac{\beta}{2}\frac{\beta+2}{4} x^2 + \Ld(x^3)\]

  \begin{IEEEeqnarray*}{rCl}
  \frac{1}{\mu} \Wt(E) &=& \tr E + \frac{1}{\beta}
    \left\{\det(C)^{-\frac{\beta}{2}} - 1 \right\} \\
  &=& \tr E - \tr E - (\tr E)^2 + \tr E^2 + \frac{\beta+2}{2} (\tr E)^2
  + \Ld(E^3) \\
  &=& \tr E^2 + \frac{\beta}{2} (\tr E)^2
  + \Ld(E^3)
  \end{IEEEeqnarray*}
  Daher sind die Lam\'{e}-Parameter $\mu = \mu, \lambda = \mu\beta$.

% ==================== 10.3 ====================
\item
  Man wähle die zwei folgenden Matrizen mit positiver Determinante:
  \[
    A = \begin{pmatrix} 1 & 0 \\ 0 & 1\end{pmatrix}, \quad
    B = \begin{pmatrix} -1 & 0 \\ 0 & -1\end{pmatrix}
  \]
  Die Konvexkombination $(1 - t) A + t B$ hat für $t = \frac{1}{2}$ Determinante
  gleich 0.
  Die Aussage lässt sich für alle $d \geq 2$ verallgemeinern, indem man die
  ersten zwei Diagonaleinträge wie oben wählt und die restliche Diagonale mit Einsern füllt.

% ==================== 10.4 ====================
\item
  \begin{enumerate}
  \item
  \item
  \end{enumerate}

% ==================== 10.5 ====================
\item

% ==================== 10.6 ====================
\item
  Zuerst mit der Definition von $\eps$, $\tau$ symmetrisch und $A:B=A^\top:B^\top$
  \[\tau : \eps(v) = \tau : \frac{1}{2} ((\nabla v)^\top + \nabla v) = \tau : \nabla v\]

  Wir wenden partielle Integration auf jede Zeile von $\tau$ an
  ($\tau_{i,\cdot}$ bezeichnet die i-te Zeile von $\tau$):
  \[ \int_\Omega v_i \divop \tau_{i,\cdot} \dx = - \int_\Omega
    \nabla v_i \cdot \tau_{i,\cdot} \dx + \int_{\partial\Omega} v_i
    (\tau_{i,\cdot} \cdot n) \ds
  \]

  Nun wird das für jede Zeile gemacht, die Zeilen addiert und man erhält die gewünschte Formel:
  \begin{IEEEeqnarray*}{rCl}
    \int_\Omega \tau : \eps(v) \dx &=& \int_\Omega \tau : \nabla v \dx
    = \sum_{i=1}^3 \int_\Omega \tau_{i,\cdot} \cdot \nabla v_i \dx \\
    &=& -\sum_{i=1}^3 \int_\Omega v_i \cdot \divop \tau_{i,\cdot} \dx
    + \int_{\partial\Omega} v_i (\tau_{i,\cdot} \cdot n) \ds \\
    &=& -\int_\Omega v \cdot \divop \tau \dx
    + \int_{\partial\Omega} v \cdot (\tau \cdot n) \ds \\
  \end{IEEEeqnarray*}

\end{enumerate}
\end{document}